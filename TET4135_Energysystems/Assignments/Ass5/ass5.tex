\documentclass{article}
\usepackage{a4wide}
\usepackage{german}
\usepackage{amsmath}
\usepackage{amssymb}
\usepackage{dsfont}
%\usepackage[dvips]{epsfig}
%\usepackage{graphicx}
\usepackage{fancyhdr}
\usepackage{listings}
\usepackage{nomencl}
\usepackage{caption}
\usepackage[pdftex]{graphicx}

\pagestyle{fancy}
\lhead{\footnotesize \parbox{11cm}{Andreas Johann H\"ormer (753179)}}
\rhead{\footnotesize {Assignment 5}}
\chead{\footnotesize {TET4135}}

\begin{document}
\section{Water value}
The water value is the expected value of stored MWh depending on capacities and inflow\footnote{Doorman Gerard et al, \textit{Kompendium TET4135 Energy Systems Planning and Operation}, p.90}. If the water value in a reservoir is known, the value euals the value the additional water will decrease the marginal costs when it is stored.
\section{Calculation}
The calculation for water values for three different inflow (10GWh, 20GWh, 30GWh) are listed in tables \ref{tab:10}, \ref{tab:20} and \ref{tab:30}. 
\begin{table}
\begin{center}
\begin{tabular}{|c|c|c|c|c|c|}
\hline
water value & inflow & hydro produced & purchase & final reservoir	& water value\\
    t       &  GWh   &    GWh         &     GWh       &     GWh         & t+1 \\
\hline
\hline
200 & 10 & 13.33 & 6.67 & 60+10-13.33=56.67 & 240\\
240 & 10 & 12 & 8 & 66.67-12=54.67 & 264\\
264 & 10 & 11.2 & 8.8 & 64.67-11.2=53.47 & 278\\
278 & 10 & 10.73 & 9.27 & 63.47-10.73=52.74 & 287\\
287 & 10 & 10.43 & 9.57 & 62.74-10.43=52.31 & 292.3\\
292.3 & 10 & 10.26 & 9.74 & 62.31-10.26=52.05 & 295.4\\
295.4 & 10 & 10.15 & 9.85 & 51.9 & 297.2\\
297.2 & 10 & 10.09 & 9.91 & 51.81 & 298.3\\
298.3 & 10 & 10.06 & 9.94 & 51.75 & 299\\
299 & 10 & 10.03 & 9.97 & 51.72 & 299.36\\
299.36 & 10 & 10.02 & 9.98 & 51.7 & 299.6\\
299.6 & 10 & 10.014 & 9.986 & 51.686 & 299.786\\
299.786 & 10 & 10.01 & 9.99 & 51.67 & 299.96\\
\hline
299.96 & 10 & 10 & 10 & 51.67 & 299.96\\
\hline
\end{tabular}
\caption{water value calculation for inflow 10GWh}
\label{tab:10}
\end{center}
\end{table}

\begin{table}
\begin{center}
\begin{tabular}{|c|c|c|c|c|c|}
\hline
water value & inflow & hydro produced & purchase & final reservoir	& water value\\
    t       &  GWh   &    GWh         &     GWh       &     GWh         & t+1 \\
\hline
\hline
200 & 20 & 13.33 & 6.67 & 80-13.33=66.67 & 120\\
120 & 20 & 18 & 2 & 68.67 & 113.3\\
113.3 & 20 & 18.67 & 1.33 & 70 & 100\\
\hline
100 & 20 & 20 & 0 & 70 & 100\\
\hline
\end{tabular}
\caption{water value calculation for inflow 20GWh}
\label{tab:20}
\end{center}
\end{table}

\begin{table}
\begin{center}
\begin{tabular}{|c|c|c|c|c|c|}
\hline
water value & inflow & hydro produced & purchase & final reservoir	& water value\\
    t       &  GWh   &    GWh         &     GWh       &     GWh         & t+1 \\
\hline
\hline
200 & 30 & 13.33 & 6.67 & 60+30-13.33=76.67 & 61.3\\
61.3 & 30 & 29.35 & 0 & 77.32 & 61.07\\
61.07 & 30 & 29.47 & 0 & 77.85 & 60.86\\
60.86 & 30 & 29.59 & 0 & 78.28 & 60.69\\
60.69 & 30 & 29.66 & 0 & 78.62 & 60.55\\
60.55 & 30 & 29.725 & 0 & 78.9 & 60.44\\
60.44 & 30 & 29.78 & 0 & 79.12 & 60.35\\
50.25 & 30 & 29.825 & 0 & 79.3 & 60.28\\
60.28 & 30 & 29.86 & 0 & 79.44 & 60.22\\
60.22 & 30 & 29.89 & 0 & 79.55 & 60.18\\
60.18 & 30 & 29.91 & 0 & 79.64 & 60.14\\
60.14 & 30 & 29.93 & 0 & 79.71 & 60.1\\
\hline
60.1 & 30 & 29.95 & 0 & 79.76 & 60.1\\
\hline
\end{tabular}
\caption{water value calculation for inflow 30GWh}
\label{tab:30}
\end{center}
\end{table}

$$0.2\cdot 299.96 + 0.6\cdot 100 + 0.2\¢dot 60.1 = 132.02 NOK/GWh $$
\section{das dritte}
\end{document}
