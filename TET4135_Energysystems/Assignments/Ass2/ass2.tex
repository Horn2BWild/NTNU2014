\documentclass{article}
\usepackage{a4wide}
\usepackage{german}
\usepackage{amsmath}
\usepackage{amssymb}
\usepackage{dsfont}
\usepackage[dvips]{epsfig}
\usepackage{graphicx}
\usepackage{fancyhdr}

\pagestyle{fancy}
\lhead{\footnotesize \parbox{11cm}{Andreas Johann H\"ormer (753179)}}
\rhead{\footnotesize {Assignment 2}}
\chead{\footnotesize {TET4135}}

\begin{document}
	\paragraph{Name, Studentnr: }Andreas Johann H\"ormer (753179)
	\paragraph{Assignment 2}Primary Energy Factors (Date: 09.02.2014)
	\paragraph{general calculations}
		\begin{itemize}
			\item total area $28\,000m^2$
			\item design heating load $75W/m^2$
			\item coincidence factor $0.8$
			\item heating demand full load $3000h$
			\item total annual heat extraction $30\,000kWh$
		\end{itemize}
		maximum design heating load: $75W/m^2\cdot 28\,000m^2 = 2\,100kW$\\
		total heat consumption in DH system: $2\,100\cdot 0.8\cdot 3000h = 5040GWh$\\

	\newpage
	\subsection*{Assignment 2a}
		\begin{itemize}
			\item efficiency of the heating network: $\eta_{hn}=0.9$
			\item power to heat ratio: $\sigma=0.5$
			\item $f_{p,el}=2.8$
			\item $Q_{del}=30GWh\cdot 20\%=6GWh$
			\item total annual power production\\
				$$E_{el}=\sigma \cdot 30\,000MWh = 15\,000MWh$$
			\item net power loss due to DH system\\
				$$\frac{delivered}{total}\cdot (power\ loss\ of\ plant)$$
				$$\frac{6\,000}{30\,000}\cdot (15\,000*0.2) = 600MWh/year$$
			\item primary energy factor\\
				$$f_{p,dh}=\frac{\sum_i E_{f,i}\cdot f_{p,F,i}-E_{el,chp}\cdot f_{p,el}}{\sum_j Q_{del,j}}$$
				$$f_{p,dh}=\frac{2.8\cdot (net\ power\ loss\ due\ to\ DH\ system)}{0.9\cdot (total\ heat\ consumption\ in\ DH\ system)}$$
				$$f_{p,dh}=\frac{2.8\cdot 600}{0.9\cdot 5040} = 0.37$$
		\end{itemize}

	\newpage
	\subsection*{Assignment 2b}
		\begin{itemize}
			\item boiler
				\begin{itemize}
					\item maximum power $P_{max,boiler}=1.4MW$
					\item fuel efficiency $\eta_{T,gen}=0.85$
					\item full load time $1000h$
				\end{itemize}
			\item CHP
				\begin{itemize}
					\item heat to power factor $\sigma=0.95$
					\item total efficiency $\eta_{chp}=0.72$
				\end{itemize}
			\item heating net efficiency $\eta_{hn}=0.75$
			\item $f_{p,el}=3.31$
			\item $f_{p,gas}=f_{p,T,gen}=1.3$
			\item $Q_{del}=5\,040MWh$
		\end{itemize}

		\subsubsection*{total amount of generated heating energy $Q$}
			$$Q=Q_{chp}+Q_{gas}$$
			$$Q=1\,411MWh+1\,400MWh=2\,811MWh$$
		\subsubsection*{amount of generated heating energy by gas boiler $Q_{gas}$}
			$$Q_{gas}=P_{max,boiler}\cdot duration$$
			$$Q_{gas}=1.4MW\cdot 1\,000h=1\,400MWh$$
		\subsubsection*{amount of heat delivered by CHP $Q_{chp}$}
			$$P_{max_{chp}}=\frac{0.28\cdot 5\,040MW}{2\,000h}=0.706MW$$
			$$Q_{chp}=P_{max_{chp}}\cdot duration$$
			$$Q_{chp}=0.706MW\cdot 2\,000h = 1\,411MWh$$
		\subsubsection*{ratio of CHP produced heat and total produced heat}
			$$\beta=\frac{1.4MW\cdot 1\,000h}{5\,040MWh}=0.28$$
		\subsubsection*{amount of fuel energy delivered to the CHP $Q_B$}
			$$Q_B=\frac{E_{el,chp}+Q_{chp}}{\eta_{tot}}$$
			$$Q_B=\frac{1\,340.6MWh+1\,411MWh}{0.72}=3822MWh$$
		\subsubsection*{total annual electricity production by the CHP $E_{el}$}
			$$E_{el}=Q_{chp}\cdot \sigma$$
			$$E_{el}=1\,411MWh\cdot 0.95 = 1\,340.6MWh$$
		\subsubsection*{elecricity efficiency of the CHP $\eta_{el}$}
			$$\eta_{el}=\frac{E_{el}}{Q_B}$$
			$$\eta_{el}=\frac{1\,340.6MWh}{3\,822MWh}=0.35$$
		\subsubsection*{primary energy factor $f_p$}
			$$f_{p,dh}=\frac{(1+\sigma)\cdot\beta}{\eta_{hn}\eta_{chp}}\cdot f_{p,chp}+\frac{1-\beta}{\eta_{hn}\eta_{T,gen}}\cdot f_{p,T,gen}-\frac{\sigma\beta}{\eta_{hn}}\cdot f_{p,el}$$
			$$f_{p,dh}=\frac{(1+0.95)\cdot 0.28}{0.75\cdot 0.72}\cdot 1.3+\frac{1-0.28}{0.75\cdot 0.85}\cdot 1.3-\frac{0.95\cdot 0.28}{0.75}\cdot 3.31$$
			$$f_{p,dh}=1.609\approx 1.61$$

	\newpage
	\subsection*{Assignment 2c}
		\begin{itemize}
			\item $\sigma=0.95$
			\item $\eta_{chp}=0.75$
			\item $\eta_{hn}=0.75$
			\item $f_{p,el}=3.31$
			\item $f_{p,gas}=f_{p,chp}=1.3$
		\end{itemize}

		all energy is produced by the CHP, so the ratio $\beta$ of CHP produced heat to total produced heat is $1$
		$$P_{max,chp}=\frac{5\,040MWh}{3\,000h}=1.68MW$$

		\subsubsection*{total amount of generated heating energy $Q_{chp}$}
			$$Q_{chp}=P_{max,chp}\cdot duration$$
			$$1.86MW\cdot 3\,000h=5\,040MWh$$
		\subsubsection*{amount of heat delivered by the CHP $Q_{del}$}
			$$Q_{chp}=5\,040MWh$$
		\subsubsection*{total amount of fuel energy delivered $Q_B$}
			$$Q_B=\frac{E_{el,chp}+Q_{chp}}{\eta_{tot}}$$
			$$Q_B=\frac{4\,788MWh+5\,040MWh}{0.75}=13\,104MWh$$
		\subsubsection*{total annual electricity production $E_{el}$}
			$$E_{el}=Q_{chp}\cdot \sigma$$
			$$E_{el}=5\,040MWh\cdot 0.95=4\,788MWh$$
		\subsubsection*{electricity efficiency $\eta_{el}$}
			$$\eta_{el}=\frac{E_{el}}{Q_B}$$
			$$\eta_{el}=\frac{4\,788MWh}{13\,104MWh}=0.37$$
		\subsubsection*{primary energy factor $f_p$}
			$$f_{p,dh}=\frac{(1+\sigma)\cdot\beta}{\eta_{hn}\eta_{chp}}\cdot f_{p,chp}-\frac{\sigma\beta}{\eta_{hn}}\cdot f_{p,el}$$
			$$f_{p,dh}=\frac{(1+0.95)}{0.75\cdot 0.75}\cdot 1.3-\frac{0.95}{0.75}\cdot 3.31$$
			$$f_{p,dh}\approx 0.31$$

	\newpage
	\subsection*{Assignment 2d}
		\begin{itemize}
			\item $P_{max}=1.4MW$
			\item $\eta_{T,gen}=1.1$
			\item $f_{p,gas}=f_{p,gen}=1.3$
			\item $\eta_{hn}=0.75$
		\end{itemize}

		$$\beta=\frac{1.4MW\cdot 3\,000h}{5\,040MWh}=0.83$$

		\subsubsection*{total amount of generated heating energy $Q_{gas}$}
			$$Q_{gas}=P_{max,gas}\cdot duration$$
			$$Q_{gas}=1.4MW\cdot 3\,000h = 4\,200MWh$$
		\subsubsection*{amount of heat delivered $Q_{del}$}
			$$Q_{del}=Q_{gas}\cdot \eta_{hn}$$
			$$Q_{del}=4\,200MWh\cdot 0.75 = 3\,150MWh$$
		\subsubsection*{primary energy factor}
			$$f_{p,dh}=\frac{1-\beta}{\eta_{hn}\eta_{T,gen}}\cdot f_{p,T,gen}$$
			$$f_{p,dh}=\frac{1-0.83}{0.75\cdot 1.1}\cdot 1.3$$
			$$f_{p,dh}\approx 0.26$$

	\newpage
	\subsection*{Assignment 2e}
The example in 2d gives the lowest energy factor. This is because of a high net calorific efficiency and a placement within the district heating boundaries. A good possibility to decrease primary energy factors is to deliver produced energy to the power grid. These terms are subtracted from the primary energy factor, and so this is getting lower. To decrease the primary energy factor one possibility could be to measure the heat which is not used and flow back to the power plant, get environmental factors like $CO_2$ and the geographical position of the plant, as well as the climatic data of the region.
\end{document}
