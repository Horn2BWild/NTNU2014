\documentclass{article}
\usepackage{a4wide}
\usepackage{german}
\usepackage{amsmath}
\usepackage{amssymb}
\usepackage{dsfont}
\usepackage[dvips]{epsfig}
\usepackage{graphicx}

\begin{document}
	\paragraph{Name, Studentnr: }Andreas Johann Hoermer (753179)
	\paragraph{Assignment 1}Optimal combinations of production plants - local energy planning (Date: 31.01.2014)

	\section*{Assignment 1a}
		\subsection*{peak load operation time $\tau_{PL}$}
			\paragraph{peak load: gas boiler}
				\begin{itemize}
					\item 0.1 EUR/kWh operational cost ($=r_P$)
					\item 100 EUR/kW investition cost ($=f_P$)
				\end{itemize}
			\paragraph{base load: bio fuel boiler}
				\begin{itemize}
					\item 0.05 EUR/kWh operational cost ($=r_B$)
					\item 300 EUR/kW investition cost ($=f_B$)
				\end{itemize}
			\paragraph{formulary}
			\begin{equation}
				\tau_{PL} = \frac{f_B - f_P}{r_P - r_B}
			\end{equation}
			\paragraph{calculation}
			$$\tau_{PL} = \frac{300-100}{0.1-0.05} = 4\,000h$$
		\subsection*{maximum effect base load $\.{Q}_{max,BL}$}
			\paragraph{formulary}
			\begin{equation}
				\.{Q}_{max,BL}=\.{Q}_{h}(\tau_{PL})
			\end{equation}
			\begin{equation}
				\.{Q}_{h}=\.{Q}_{h,min}+(\.{Q}_{h,max}-\.{Q}_{h,min})\cdot [1-\sqrt[3]{\frac{\tau}{\tau_{0}}}+(\frac{\tau}{\tau_{0}})^2\cdot (1-\sqrt{\frac{\tau}{\tau_{0}}})]
			\end{equation}
			\paragraph{calculation}
			$$\.{Q}_{max,BL}=\.{Q}_{h}(4\,000h)=5\,000kW+(60\,000kW-5\,000kW)\cdot [1-\sqrt[3]{\frac{4\,000h}{6\,500h}}+(\frac{4\,000h}{6\,500h})^2\cdot (1-\sqrt{\frac{4\,000h}{6\,500h}})] $$
			$$\.{Q}_{max,BL}=17\,707.3kW $$
		\subsection*{maximum effect peak load $\.{Q}_{max,PL}$}
			\paragraph{formulary}
			\begin{equation}
				\.{Q}_{max,PL}=\.{Q}_{h,max}-\.{Q}_{max,BL})
			\end{equation}
			\paragraph{calculation}
			$$\.{Q}_{max,PL}=60\,000kW-17\,707.3kW = 42\,292.7kW$$
		\subsection*{Energy use}
			\paragraph{formulary}
			\begin{equation}
				Q=\int{\.{Q}d\tau}
			\end{equation}
			\begin{equation}
				Q_{PL}=\int_{0}^{\tau_{PL}}{\.{Q}_{h}d\tau}-\.{Q}_{h}(\tau_{PL})\cdot \tau_{PL}
			\end{equation}
			\begin{equation}
				\int_{0}^{\tau_{PL}}{\.{Q}_{h}d\tau}=\.{Q}_{h,min}\cdot\tau |_{0}^{\tau_{PL}}+
												\Delta\.{Q}_{h}\cdot [\tau |_{0}^{\tau_{PL}}-
												\frac{1}{\sqrt[3]{\tau_{0}}}\cdot\frac{\tau^{\frac{4}{3}}}{\frac{4}{3}}|_{0}^{\tau_{PL}}+
												\frac{1}{\tau_{0}^2}\cdot\frac{\tau^3}{3}|_{0}^{\tau_{PL}}-
												\frac{1}{\tau_{0}^{\frac{5}{2}}}\cdot\frac{\tau^{\frac{7}{2}}}{\frac{7}{2}} |_{0}^{\tau_{PL}}]
			\end{equation}
			\begin{equation}
				Q_{PL}=\int_{0}^{\tau_{PL}}{\.{Q}_{h}d\tau}-\.Q_{h}(\tau_{PL})\cdot\tau_{PL}
			\end{equation}
			\begin{equation}
				Q_{BL}=\.Q(\tau_{PL})\cdot \tau_{PL}+\.Q_{DHW}(\tau_{year}-\tau_{PL})+\int_{\tau_{PL}}^{\tau_{0}}{\.{Q}_{h}d\tau}
			\end{equation}
			\paragraph{calculation}
			$$\int_{0}^{4000}{\.{Q}_{h}d\tau}=5\,000\cdot 4\,000+
												(60\,000-5\,000)\cdot [4\,000-
												\frac{1}{\sqrt[3]{\,6500}}\cdot\frac{4\,000^{\frac{4}{3}}}{\frac{4}{3}}+
												\frac{1}{6\,500^2}\cdot\frac{4\,000^3}{3}-
												\frac{1}{6\,500^{\frac{5}{2}}}\cdot\frac{4\,000^{\frac{7}{2}}}{\frac{7}{2}}]$$
			$$\int_{0}^{4000}{\.{Q}_{h}d\tau}=108\,752\,075kWh$$
			$$Q_{PL}=108\,752\,075kWh-17\,707.3kW\cdot 4\,000h = 37\,922\,875kWh$$
			$$Q_{BL}=17\,707.3\cdot 4\,000 + 5\,000\cdot(8\,760-4\,000)+5\,000\cdot (6\,500-4\,000)+55\,000[(6\,500-4\,000)-$$
												$$\frac{1}{\sqrt[3]{6\,500}}\cdot\frac{6\,500^{\frac{4}{3}}-4\,000^{\frac{4}{3}}}{\frac{4}{3}}+
												\frac{1}{6\,500^2}\cdot\frac{6\,500^3-4\,000^3}{3}-
												\frac{1}{6\,500^{\frac{5}{2}}}\cdot\frac{6\,500^{\frac{7}{2}}-4\,000^{\frac{7}{2}}}{\frac{7}{2}}]$$
			$$Q_{BL}=132\,775\,935kWh$$
		\subsection*{total annual cost}
			\paragraph{formulary}
			\begin{equation}
				C = c_{gas}\cdot Q_{PL}+c_{bio}\cdot Q_{BL}
			\end{equation}
			\paragraph{calculation}
				$$C=37\,922\,875kWh\cdot 0.1 ^{EUR}/_{kWh} + 132\,775\,935kWh\cdot 0.05 ^{EUR}/_{kWh}$$
				$$C=10\,431\,085EUR$$












\newpage

\section*{Assignment 1b}
		\subsection*{peak load operation time $\tau_{PL}$}
			\paragraph{peak load: gas boiler}
				\begin{itemize}
					\item 0.1 EUR/kWh operational cost ($=r_P$)
					\item 130 EUR/kW investition cost ($=f_P$)
				\end{itemize}
			\paragraph{base load: waste combustion boiler}
				\begin{itemize}
					\item -0.08 EUR/kWh operational cost ($=r_B$)
					\item 1300 EUR/kW investition cost ($=f_B$)
				\end{itemize}

			$$\tau_{PL} = \frac{1300-130}{0.1+0.08} = 6\,500h$$
		\subsection*{maximum effect base load $\.{Q}_{max,BL}$}
			$$\.{Q}_{max,BL}=\.{Q}_{h}(6\,500h)=5\,000kW+(60\,000kW-5\,000kW)\cdot [1-\sqrt[3]{\frac{6\,500h}{6\,500h}}+(\frac{6\,500h}{6\,500h})^2\cdot (1-\sqrt{\frac{6\,500h}{6\,500h}})] $$
			$$\.{Q}_{max,BL}=5\,000kW $$
		\subsection*{maximum effect peak load $\.{Q}_{max,PL}$}
			$$\.{Q}_{max,PL}=60\,000kW-5\,000kW = 55\,000kW$$
		\subsection*{Energy use}
			$$\int_{0}^{6500}{\.{Q}_{h}d\tau}=5\,000\cdot 6\,500+
												55\,000\cdot [6\,500-
												\frac{1}{\sqrt[3]{\,6500}}\cdot\frac{6\,500^{\frac{4}{3}}}{\frac{4}{3}}+
												\frac{1}{6\,500^2}\cdot\frac{6\,500^3}{3}-
												\frac{1}{6\,500^{\frac{5}{2}}}\cdot\frac{6\,500^{\frac{7}{2}}}{\frac{7}{2}}]$$
			$$\int_{0}^{6500}{\.{Q}_{h}d\tau}=138\,898\,809kWh$$
			$$Q_{PL}=138\,898\,809kWh-5\,000kW\cdot 6\,500h = 106\,398\,809kWh$$
			$$Q_{BL}=5\,000\cdot 6\,500+(8\,760-6\,500)\cdot 5\,000+\int_{6500}^{6500}{...}$$
			$$Q_{BL}=43\,000\,000kWh$$
		\subsection*{total annual cost}
			\paragraph{formulary}
			\begin{equation}
				C = c_{gas}\cdot Q_{PL}+c_{waste}\cdot Q_{BL}
			\end{equation}
			\paragraph{calculation}
				$$C=138\,898\,809kWh\cdot 0.1 ^{EUR}/_{kWh} + 43\,800\,000kWh\cdot (-0.08 ^{EUR}/_{kWh})$$
				$$C=10\,385\,881EUR$$




\newpage

	\section*{Assignment 1c}
The annual costs are pretty much the same. In 1a the peak load operation time is 4000h, so the sizes of the two boilers are about 1:3. This is because of three times investment costs and half energy prices. In 1b the waste combustion boiler is only used to provide the base load. The waste combustions creates profit, but the investment is quite high. So a investment for providing energy which is not used for the full time (peak load) is not useful. The use of gas boilers for peak loads is usefful, because the investment costs for huge amounts of energy are quite cheap. Bio-fuel for base load is also quite cheap, so that it could be used for base load and some kind of peak load. Waste combustion is only useful for base load because of investment costs. 





\end{document}
