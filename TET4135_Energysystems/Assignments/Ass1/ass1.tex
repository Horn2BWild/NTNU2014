\documentclass{article}
\usepackage{a4wide}
\usepackage{german}
\usepackage{amsmath}
\usepackage{amssymb}
\usepackage{dsfont}
\usepackage[dvips]{epsfig}
\usepackage{graphicx}

\begin{document}
	\paragraph{Name, Studentnr: }Andreas Johann Hoermer ()
	\paragraph{Assignment 1}Optimal combinations of production plants - local energy planning (Date: 31.01.2014)

	\section*{Assignment 1a}
		\subsection*{peak load operation time $\tau_{PL}$}
			\paragraph{peak load: gas boiler}
				\begin{itemize}
					\item 0.1 EUR/kWh operational cost ($=r_P$)
					\item 100 EUR/kW investition cost ($=f_P$)
				\end{itemize}
			\paragraph{base load: bio fuel boiler}
				\begin{itemize}
					\item 0.05 EUR/kWh operational cost ($=r_B$)
					\item 300 EUR/kW investition cost ($=f_B$)
				\end{itemize}
			\paragraph{formulary}
			\begin{equation}
				\tau_{PL} = \frac{f_B - f_P}{r_P - r_B}
			\end{equation}
			\paragraph{calculation}
			$$\tau_{PL} = \frac{300-100}{0.1-0.05} = 4000h$$
		\subsection*{maximum effect base load $\.{Q}_{max,BL}$}
			\paragraph{formulary}
			\begin{equation}
				\.{Q}_{max,BL}=\.{Q}_{h}(\tau_{PL})
			\end{equation}
			\begin{equation}
				\.{Q}_{h}=\.{Q}_{h,min}+(\.{Q}_{h,max}-\.{Q}_{h,min})\cdot [1-\sqrt[3]{\frac{\tau}{\tau_{0}}}+(\frac{\tau}{\tau_{0}})^2\cdot (1-\sqrt{\frac{\tau}{\tau_{0}}})]
			\end{equation}
			\paragraph{calculation}
			$$\.{Q}_{max,BL}=\.{Q}_{h}(4000h)=5000kW+(60000kW-5000kW)\cdot [1-\sqrt[3]{\frac{4000h}{6500h}}+(\frac{4000h}{6500h})^2\cdot (1-\sqrt{\frac{4000h}{6500h}})] $$
			$$\.{Q}_{max,BL}=17707.3kW $$
		\subsection*{maximum effect peak load $\.{Q}_{max,PL}$}
			\paragraph{formulary}
			\begin{equation}
				\.{Q}_{max,PL}=\.{Q}_{h,max}-\.{Q}_{max,BL})
			\end{equation}
			\paragraph{calculation}
			$$\.{Q}_{max,PL}=60000kW-17707.3kW = 42292.7kW$$
		\subsection*{Energy use}
			\paragraph{formulary}
			\begin{equation}
				Q=\int{\.{Q}d\tau}
			\end{equation}
			\begin{equation}
				Q_{PL}=\int_{0}^{\tau_{PL}}{\.{Q}_{h}d\tau}-\.{Q}_{h}(\tau_{PL})\cdot \tau_{PL}
			\end{equation}
			\begin{equation}
				\int_{0}^{\tau_{PL}}{\.{Q}_{h}d\tau}=\.{Q}_{h,min}\cdot\tau |_{0}^{\tau_{PL}}+
												\Delta\.{Q}_{h}\cdot [\tau |_{0}^{\tau_{PL}}-
												\frac{1}{\sqrt[3]{\tau_{0}}}\cdot\frac{\tau^{\frac{4}{3}}}{\frac{4}{3}}|_{0}^{\tau_{PL}}+
												\frac{1}{\tau_{0}^2}\cdot\frac{\tau^3}{3}|_{0}^{\tau_{PL}}-
												\frac{1}{\tau_{0}^{\frac{5}{2}}}\cdot\frac{\tau^{\frac{7}{2}}}{\frac{7}{2}} |_{0}^{\tau_{PL}}]
			\end{equation}
			\begin{equation}
				Q_{PL}=\int_{0}^{\tau_{PL}}{\.{Q}_{h}d\tau}-\.Q_{h}(\tau_{PL})\cdot\tau_{PL}
			\end{equation}
			\begin{equation}
				Q_{BL}=\.Q(\tau_{PL})\cdot \tau_{PL}+\.Q_{DHW}(\tau_{year}-\tau_{PL})+\int_{\tau_{PL}}^{\tau_{0}}{\.{Q}_{h}d\tau}
			\end{equation}
			\paragraph{calculation}
			$$\int_{0}^{4000}{\.{Q}_{h}d\tau}=5000\cdot 4000+
												(60000-5000)\cdot [4000-
												\frac{1}{\sqrt[3]{6500}}\cdot\frac{4000^{\frac{4}{3}}}{\frac{4}{3}}+
												\frac{1}{6500^2}\cdot\frac{4000^3}{3}-
												\frac{1}{6500^{\frac{5}{2}}}\cdot\frac{4000^{\frac{7}{2}}}{\frac{7}{2}}]$$
			$$\int_{0}^{4000}{\.{Q}_{h}d\tau}=108752075kWh$$
			$$Q_{PL}=108752075kWh-17707.3kW\cdot 4000h = 37922875kWh$$
		\subsection*{total annual cost}
\end{document}
