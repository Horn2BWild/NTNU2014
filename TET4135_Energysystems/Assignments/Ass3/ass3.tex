\documentclass{article}
\usepackage{a4wide}
\usepackage{norsk}
\usepackage{amsmath}
\usepackage{amssymb}
\usepackage{dsfont}
%\usepackage[dvips]{epsfig}
%\usepackage{graphicx}
\usepackage{fancyhdr}
\usepackage{listings}
\usepackage{nomencl}
\usepackage[pdftex]{graphicx}

\usepackage{float}
\restylefloat{table}

\pagestyle{fancy}
\lhead{\footnotesize \parbox{11cm}{Andreas Johann H\"ormer (753179)}}
\rhead{\footnotesize {Assignment 3}}
\chead{\footnotesize {TET4135}}

\begin{document}
%	\paragraph{Name, Studentnr: }Andreas H\"ormer (753179)
%	\paragraph{Assignment 3}Optimal generation dispatch (Date: 18.02.2014)
	\section*{General}
		\subsection*{power station 1}
			Costs of starting and stoping are assumed negligible ($=0$).
			\begin{itemize}
				\item $K_1=500+45\cdot P_1+0.075\cdot P_1^2 [EUR/h]$
				\item $P_{1_{min}}=20MW$
				\item $P_{1_{max}}=100MW$
			\end{itemize}
		\subsection*{power station 2}
			\begin{itemize}
				\item $K_2=750+52\cdot P_2+0.025\cdot P_2^2 [EUR/h]$
				\item $P_{2_{min}}=40MW$
				\item $P_{2_{max}}=200MW$
			\end{itemize}
	\section{optimal operation of both stations}
		minimum costs, where K1 and K2 intersect ($K_1=K_2$). This is at a production value of $P=169.5MW$
	%Grafik
	\section{Production alternatives}
	\subsection{operation possibilities}
In table \ref{tab:oppos} the possibilities for plant combination at different production intervals is listed.
	\begin{table}[h!]
\begin{center}
\begin{tabular}[h]{|c|c|}
\hline 
Interval (MW) & power station\\ \hline
20..40 & $K_1$\\
40..60 & $K_1$, $K_2$\\
60..100 & $K_1$, $K_2$, $K_1+K_2$\\
100..200 & $K_2$, $K_1+K_2$\\
200..300 & $K_1+K_2$\\
\hline
\end{tabular}
\caption{Operation possibilities for different production}\label{tab:oppos}
\end{center}
\end{table}

\subsection{cost function}
	\begin{equation}
		K_1=500+45\cdot P_1 + 0.075 P_1^2
		\label{f:K1}
	\end{equation}
	\begin{equation}
		K_2=750+52\cdot P_2 + 0.025 P_2^2
		\label{f:K2}
	\end{equation}
The equations \ref{f:K1} and \ref{f:K2} can be derived and the equal marginal costs calculated.
	\begin{equation}
		\frac{dK_1}{dP_1}=45+0.15\cdot P_1
		\label{f:dK1}
	\end{equation}
	\begin{equation}
		\frac{dK_2}{dP_2}=52+0.05\cdot P_2
		\label{f:dK2}
	\end{equation}
	The total costs are minimal where equations \ref{f:dK1} and \ref{f:dK2} are the same. For production of less than 40MW the operation of plant 1 only is possible. So the cost function is not listed seperately.
	\subsubsection{interval 40MW-60MW}
		The Lagrange function can be written as
		\begin{equation}
			L=-(K_1+K_2)+\lambda(P_1+P_2-P_{max})
	 		\label{f:lagrange}
		\end{equation}
In this formulary \ref{f:lagrange} the cost functions for the two generation units can be filled in and the Lagrange function be derived afterwards.
		$$L=-(500+45\cdot P_1+0.075\cdot P_1^2+750+52\cdot P_2+0.025\cdot P_2^2+\lambda(P_1+P_2-60)$$
		$$L=-(1250+45\cdot P_1 + 52\cdot P_2 + 0.075\cdot P_1^2+0.025\cdot P_2^2)+\lambda(P_1+P_2-60)$$
		$$\frac{dL}{dP_1}=-(45+0.15P_1)+\lambda=0$$
		$$\frac{dL}{dP_2}=-(52+0.05P_2)+\lambda=0$$
With this derivates the optimal production amount for $P_1$ and $P_2$ can be calculated.
		$$45+0.15P_1=52+0.05P_2$$
		$$P_1=\frac{1}{3}(P_2+140)$$
	Now the constraint $P_1+P_2=60$ can be filled in.
		$$\frac{1}{3}(P_2+140)+P_2=60$$
		$$P_2=10$$
		This value is the optimum, but it is not ok due to missmatching the constraint. So the production level has to be set to the smallest possible value $P_2$ can procuce for 2-station operation. The production therefore is
		\begin{itemize}
			\item $P_2=P_{2_{min}}=40MW$
			\item $P_1=60MW-40MW=20MW$
		\end{itemize}
		Both values are meeting with the constraints, so this is the optimal solution for two station operation. The costs are calculated as follows:
		\begin{itemize}
			\item $K_1$ only: $K_1(60)=500+45\cdot 60+0.075\cdot 60^2=3470EUR$
			\item $K_2$ only: $K_2(60)=750+52\cdot 60+0.025\cdot 60^2=3960EUR$
			\item $K_1+K_2$: $K=1250+45\cdot 20+0.075\cdot 20^2 + 52\cdot 40+0.025\cdot 40^2=4300EUR$
		\end{itemize}
	\subsubsection{interval 60MW-100MW}
This interval is calculated using tables and increasing the production amount with regard to the actual costs. The calculation steps are listed in table \ref{tab:60100MW}.
	\begin{table}[h!]
		\begin{center}
			\begin{tabular}[h]{|r||c|c|c||p{5cm}|}
				\hline 
				($P_1$,$P_2$) [MW] & $\frac{dK_1}{dP_1}$ & $\frac{dK_2}{dP_2}$ & difference & comment\\
				\hline
				\hline
				(0,0)	&	45	&	52	&	$\Delta P_1=+20$	&	costs without production									\\
						&    	&   	& 	$\Delta P_2=+40$ 	&					    										\\
				\hline
				(20,40) &	48	&	54	&	$\Delta P_1=+20$	&	lower production limit, increasing $P_1$ due to lower costs	\\
				\hline
				(40,40) &	51	&	54	&	$\Delta P_1=+20$	&	increase $P_1$, costs are still lower						\\
				\hline
				(60,40) &	54	&	54	&						&	costs are equal, sum of production is 100MW					\\
				\hline
			\end{tabular}
			\caption{Calculation of production amount in interval 60MW-100MW}\label{tab:60100MW}
		\end{center}
	\end{table}
\end{document}
