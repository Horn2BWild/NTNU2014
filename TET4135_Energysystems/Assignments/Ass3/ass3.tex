\documentclass{article}
\usepackage{a4wide}
\usepackage{norsk}
\usepackage{amsmath}
\usepackage{amssymb}
\usepackage{dsfont}
%\usepackage[dvips]{epsfig}
%\usepackage{graphicx}
\usepackage{fancyhdr}
\usepackage{listings}
\usepackage{nomencl}
\usepackage[pdftex]{graphicx}

\pagestyle{fancy}
\lhead{\footnotesize \parbox{11cm}{Andreas Johann H\"ormer (753179)}}
\rhead{\footnotesize {Assignment 3}}
\chead{\footnotesize {TET4135}}

\begin{document}
	\paragraph{Name, Studentnr: }Andreas H\"ormer (753179)
	\paragraph{Assignment 3}Optimal generation dispatch (Date: 18.02.2014)
	\section{General}
		\subsection{power station 1}
			Costs of starting and stoping are assumed negligible ($=0$).
			\begin{itemize}
				\item $K_1=500+45\cdot P_1+0.075\cdot P_1^2 [EUR/h]$
				\item $P_{1_{min}}=20MW$
				\item $P_{1_{max}}=100MW$
			\end{itemize}
		\subsection{power station 2}
			\begin{itemize}
				\item $K_2=750+52\cdot P_2+0.025\cdot P_2^2 [EUR/h]$
				\item $P_{2_{min}}=40MW$
				\item $P_{2_{max}}=200MW$
			\end{itemize}
	\section{optimal operation of both stations}
		minimum costs, where K1 and K2 intersect ($K_1=K_2$). This is at a production value of $P=169.5MW$
	%Grafik
	\section{Production alternatives}
	\subsection{operation possibilities}
In table \ref{tab:oppos} the possibilities for plant combination at different production intervals is listed.
	\begin{table}
\begin{center}
\begin{tabular}{|c|c|}
\hline 
Interval (MW) & power station\\ \hline
20..40 & $K_1$\\
40..60 & $K_1$, $K_2$\\
60..100 & $K_1$, $K_2$, $K_1+K_2$\\
100..200 & $K_2$, $K_1+K_2$\\
200..300 & $K_1+K_2$\\
\hline
\end{tabular}
\caption{Operation possibilities for different production}\label{tab:oppos}
\end{center}
\end{table}

\subsection{cost function}
	\begin{equation}
		K_1=500+45\cdot P_1 + 0.075 P_1^2
		\label{f:K1}
	\end{equation}
	\begin{equation}
		K_2=750+52\cdot P_2 + 0.025 P_2^2
		\label{f:K2}
	\end{equation}
The equations \ref{f:K1} and \ref{f:K2} can be derived and the equal marginal costs calculated.
	\begin{equation}
		\frac{dK_1}{dP_1}=45+0.15\cdot P_1
		\label{f:dK1}
	\end{equation}
	\begin{equation}
		\frac{dK_2}{dP_2}=52+0.05\cdot P_2
		\label{f:dK2}
	\end{equation}
	The total costs are minimal where \ref{f:dK1} and \ref{f:dK2} are the same. For production of less than 40MW the operation of plant 1 only is possible. So the cost function is not listed seperately.
	\subsubsection{interval 40MW-60MW}
\end{document}
