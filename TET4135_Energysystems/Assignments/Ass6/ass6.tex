\documentclass{article}
\usepackage{a4wide}
\usepackage{german}
\usepackage{amsmath}
\usepackage{amssymb}
\usepackage{dsfont}
\usepackage[dvips]{epsfig}
\usepackage{graphicx}

\begin{document}
\section{1}
Water from reservoirs 1 and 2 can be used for production of 3, and 1 and 2 are higher than 3. So water in the upper reservoirs has a higher water value.
\section{2}
\begin{equation}
min_{q,t}(wv_1\cdot f(\sum_{t=1}^4(l_t\cdot q_{1,t})+\sum_{t=1}^4(l_t\cdot q_{2,t}))+wv_2\cdot \sum_{t=1}^4(l_t\cdot q_{3,t}))
\end{equation}
\begin{equation}
1\frac{m^3}{s}=1\cdot 10^{-6}\frac{Mm^3}{s}=3600\cdot 10^{-6}\frac{Mm^3}{h}=3.6\cdot 10^{-3}\frac{Mm^3}{h}
\end{equation}
\begin{equation}
f=3.6\cdot 10^{-3}\frac{Ms}{h}
\end{equation}
\section{3}
Generally the values can be calculated using following formulas.
\begin{equation}
r_{1,in}=f\int_{t_0}^{t}(Q_1-s_1)dt
\end{equation}
\begin{equation}
r_{2,in}=f\int_{t_0}^{t}(Q_2-s_2)dt
\end{equation}
\begin{equation}
r_{3,in}=\int_{t_0}^{t}(q_1+s_1+q_2-s_2+Q_3)dt
\end{equation}
\begin{equation}
r_{1}=r_{1,0}+f\int_{t_0}^{t}(r_{1,in}-r{1,out})dt = r_{1,0}+f\int_{t_0}^{t}(Q_1-s_1-q_1)dt
\end{equation}
\begin{equation}
r_{2}=r_{2,0}+f\int_{t_0}^{t}(r_{2,in}-r{2,out})dt = r_{2,0}+f\int_{t_0}^{t}(Q_2-s_2-q_2)dt
\end{equation}
\begin{equation}
r_{3}=r_{3,0}+f\int_{t_0}^{t}(Q_3+q_1+s_1+q_2-s_3-q_3)dt = r_{2,0}+f\int_{t_0}^{t}(Q_2-s_2-q_2)dt
\end{equation}
Using the time intervals given the formulas change to
\begin{equation}
r_{1,t}=r_{1,t-1}+f\cdot l_t(Q_{1,t}-s_{1,t}-q_{1,t})
\end{equation}
\begin{equation}
r_{2,t}=r_{2,t-1}+f\cdot l_t(Q_{2,t}-s_{2,t}-q_{2,t})
\end{equation}
\begin{equation}
r_{3,t}=r_{3,t-1}+f\cdot l_t(Q_{3,t}+q_{1,t}+s_{1,t}+q_{2,t}-s_{3,t}-q_[3,t})
In the first time period the value of $q_2$ is not known. So the initial value for $q_{2,0} = 0$.
\end{equation}
\section{4}

\section{5}

\section{6}

\section{7}
\end{document}
