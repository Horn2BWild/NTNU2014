\documentclass{article}
\usepackage{a4wide}
\usepackage{german}
\usepackage{amsmath}
\usepackage{amssymb}
\usepackage{dsfont}
%\usepackage[dvips]{epsfig}
%\usepackage{graphicx}
\usepackage{fancyhdr}
\usepackage{listings}
\usepackage{nomencl}
\usepackage[pdftex]{graphicx}

\pagestyle{fancy}
\lhead{\footnotesize \parbox{11cm}{Andreas Johann H\"ormer (753179)}}
\rhead{\footnotesize {Laboratory 1}}
\chead{\footnotesize {TTT4170}}

\begin{document}
	\paragraph{Name, Studentnr: }Andreas H\"ormer (753179)
	\paragraph{Lab 1}Room acoustics (Date: 31.01.2014)

	\section{Theoretical part}
		\subsection{Measuring an impulse response with a swept-sine method}
			\begin{figure}[htbp]
				\begin{center}
					\includegraphics[width=8cm,keepaspectratio=true]{ir}
					\caption{Impulse response with early reflections and reverbation}
					\label{irtheory}
				\end{center}
			\end{figure}

		\subsection{Reverbation time in an auditorium}
			\subsubsection{Formulary}
				\begin{equation}
					V_{room,total}=V_{room}-V_{under\ shaped\ area}
				\end{equation}
				\begin{equation}
					V=l\cdot h\cdot w
				\end{equation}
				\begin{equation}
					\alpha=\frac{0.161\cdot V}{T_{60}\cdot S}
				\end{equation}
				\begin{equation}
					T_{60}=\frac{k\cdot V}{A}=\frac{0.161\cdot V}{\alpha\cdot S}
				\end{equation}
				\begin{equation}
					S_{total}=S_{front}+2\cdot S_{side}+S_{back}-2\cdot S_{side\ below\ slope}+S_{ceiling}+S_{floor}+S_{slope}
				\end{equation}
				\begin{equation}
					\frac{1}{4\pi\cdot r^2}=\frac{4}{A}
				\end{equation}
				\begin{equation}
					r_H=\sqrt{\frac{V}{100\cdot\pi\cdot T_{60}}}
				\end{equation}
				\begin{equation}
					\bar{\alpha}=\frac{\alpha_1\cdot S_1+\alpha_2\cdot S_2}{S_1+S_2}
				\end{equation}
			\subsubsection{Calculation}
			\paragraph{geometric calculations}
			$$V=12m\cdot 6m\cdot 15m - \frac{11m\cdot 12m\cdot 3m}{2}$$
			$$V=882m^3$$

			$$l_{slope}=\sqrt{(11m)^2+(3m)^2}=11.4m$$
			$$S=12m\cdot 6m+2\cdot 15m\cdot 6m+3m\cdot 12m-2\cdot\frac{3m\cdot 11m}{2}+12m\cdot 15m+12m\cdot 4m+11.4m\cdot 12m$$
			$$S=619.8m^2$$


			\paragraph{a) calculation of $\alpha$}
			$$\bar{\alpha}=\frac{0.161\cdot 882m^3}{619.8m^2\cdot 1s}$$
			$$\alpha=0.229$$
			
			\paragraph{b) calculation of absorption factor where the chairs are}
				$$\alpha=\frac{0.229\cdot 619.8m^2-0.1\cdot 483m^2}{136.8m^2}$$
				$$\alpha=0.684$$
			\paragraph{c) hall radius}
				$$r_H=\sqrt{\frac{882}{1s\cdot 100\cdot \pi}}=1.68m$$
		\subsection{More accurate}
			\subsubsection{Formulary}
				\begin{equation}
					A=S\cdot\bar{\alpha}+4\cdot m\cdot V
				\end{equation}
				\begin{equation}
					m\approx\frac{0.074}{\Phi}\cdot f^2
				\end{equation}
				where 
				\begin{itemize}
					\item $\Phi$ is the relative humidity
					\item $f$ is frequency in kHz
				\end{itemize}
				\begin{equation}
					\bar{\alpha}=\frac{0.161\cdot V-(S-S_{grey})\cdot\alpha-4\cdot m\cdot V}{S_{grey}}
				\end{equation}
			\subsubsection{Calculation}
				\paragraph{$f=2kHz$}
					$$\bar{\alpha}=\frac{0.161\cdot 882m^3-483m^2\cdot 0.229-4\cdot \frac{0.074}{0.5}\cdot 2^2\cdot 882m^3}{136.8m^2}$$
					$$\bar{\alpha}=$$
				\paragraph{$f=4kHz$}
					$$\bar{\alpha}=\frac{0.161\cdot 882m^3-483m^2\cdot 0.229-4\cdot \frac{0.074}{0.5}\cdot 4^2\cdot 882m^3}{136.8m^2}$$
					$$\bar{\alpha}=$$
	\section{Practical part}
\end{document}
