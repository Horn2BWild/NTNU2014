\documentclass{article}
\usepackage{a4wide}
\usepackage{german}
\usepackage{amsmath}
\usepackage{amssymb}
\usepackage{dsfont}
%\usepackage[dvips]{epsfig}
%\usepackage{graphicx}
\usepackage{fancyhdr}
\usepackage{listings}
\usepackage{nomencl}
\usepackage[pdftex]{graphicx}

\pagestyle{fancy}
\lhead{\footnotesize \parbox{11cm}{Andreas Johann H\"ormer (753179), Danny Tormoen}}
\rhead{\footnotesize {Assignment 6}}
\chead{\footnotesize {TTT4170}}

\begin{document}
\section{Problem description}
The implementation shall solve the two-dimensional poisson problem 
\begin{equation}
-\bigtriangledown^2=f in \Omega = (0,1)x(0,1)
\end{equation}
\begin{equation}
u = 0 on \delta\Omega
\end{equation}
The given poisson equation is an elliptic partial differential equation. The problem is solving this equation with given boundary conditions. 
The problem should be discretized with (n+1) points in each spatial direction. The standard 5-point-stencil to discretize the Laplace operator $\bigtriangledown$ has to be used. To obtain a solution in $O(n\cdot log(n))$ the DST should be applied.
\section{Solution strategies}
\subsection{LU factorization}
LU (lower upper) factorization decomposes a matrix in an upper and a lower triangular matrix. This can be done using Gaussian elimination. For symmetric matrizes this can also be done using the cholesky decomposition. The matrix decomposition follows following rule
\begin{equation}
A=LU
\end{equation}
where A is the original matrix, L the lower triangular matrix and U the upper triangular matrix.\\
This solution strategy is not very usable as solver in parallel contexts. The sequential nature (pivoting, substitution, ...) makes ithard for parralelization.
\subsection{cholesky factorization}

\subsection{diagonalization}
\subsubsection{diagonalization using DST}
\paragraph{row based matrix partitioning}
\paragraph{column based matrix partitioning}
\section{Implemented solution}
\subsection{matrix transpose}
\section{Environment}
\subsection{Supercomputer}
The used Kongull cluster is a CentOS 5.3 Linux cluster. The cluster has 1 login, 4 I/O and 93 compute nodes. Each node is equipped with 2x 6-core processors, with 6x 512KiB L1 cache and a common 6 MiB L3-cache. Kongull has 96 compute nodes and 1152 cores in total. All of the compute nodes are HP DL165 G6 servers, with
\begin{itemize}
\item 2 AMD Opteron model 2431 6-core (Istanbul) processors
\item 2.4 GHz core speed
\item 667 MHz (48 GiB nodes) or 800 MHz (24 GiB nodes) bus frequency
\item 149GiB 15000 RPM SAS system disc
\end{itemize}
This and further information about the used supercomputer can be found at the NTNU HPC homepage\footnote{https://www.hpc.ntnu.no/display/hpc/Kongull+Hardware, accessed: 18.03.2014}.
\subsection{Compiler}

For our solution we used following packages on kongull:
\begin{itemize}
\item intelcomp
\item openmpi
\end{itemize}
\section{Results}

\section{Analysis}

\section{Possible optimizations}

\section{Appendix}
\subsection{Code samples}
\end{document}
